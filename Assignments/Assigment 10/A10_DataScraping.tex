\PassOptionsToPackage{unicode=true}{hyperref} % options for packages loaded elsewhere
\PassOptionsToPackage{hyphens}{url}
%
\documentclass[]{article}
\usepackage{lmodern}
\usepackage{amssymb,amsmath}
\usepackage{ifxetex,ifluatex}
\usepackage{fixltx2e} % provides \textsubscript
\ifnum 0\ifxetex 1\fi\ifluatex 1\fi=0 % if pdftex
  \usepackage[T1]{fontenc}
  \usepackage[utf8]{inputenc}
  \usepackage{textcomp} % provides euro and other symbols
\else % if luatex or xelatex
  \usepackage{unicode-math}
  \defaultfontfeatures{Ligatures=TeX,Scale=MatchLowercase}
\fi
% use upquote if available, for straight quotes in verbatim environments
\IfFileExists{upquote.sty}{\usepackage{upquote}}{}
% use microtype if available
\IfFileExists{microtype.sty}{%
\usepackage[]{microtype}
\UseMicrotypeSet[protrusion]{basicmath} % disable protrusion for tt fonts
}{}
\IfFileExists{parskip.sty}{%
\usepackage{parskip}
}{% else
\setlength{\parindent}{0pt}
\setlength{\parskip}{6pt plus 2pt minus 1pt}
}
\usepackage{hyperref}
\hypersetup{
            pdftitle={Assignment 10: Data Scraping},
            pdfauthor={Rachel Gonsenhauser},
            pdfborder={0 0 0},
            breaklinks=true}
\urlstyle{same}  % don't use monospace font for urls
\usepackage[margin=2.54cm]{geometry}
\usepackage{color}
\usepackage{fancyvrb}
\newcommand{\VerbBar}{|}
\newcommand{\VERB}{\Verb[commandchars=\\\{\}]}
\DefineVerbatimEnvironment{Highlighting}{Verbatim}{commandchars=\\\{\}}
% Add ',fontsize=\small' for more characters per line
\usepackage{framed}
\definecolor{shadecolor}{RGB}{248,248,248}
\newenvironment{Shaded}{\begin{snugshade}}{\end{snugshade}}
\newcommand{\AlertTok}[1]{\textcolor[rgb]{0.94,0.16,0.16}{#1}}
\newcommand{\AnnotationTok}[1]{\textcolor[rgb]{0.56,0.35,0.01}{\textbf{\textit{#1}}}}
\newcommand{\AttributeTok}[1]{\textcolor[rgb]{0.77,0.63,0.00}{#1}}
\newcommand{\BaseNTok}[1]{\textcolor[rgb]{0.00,0.00,0.81}{#1}}
\newcommand{\BuiltInTok}[1]{#1}
\newcommand{\CharTok}[1]{\textcolor[rgb]{0.31,0.60,0.02}{#1}}
\newcommand{\CommentTok}[1]{\textcolor[rgb]{0.56,0.35,0.01}{\textit{#1}}}
\newcommand{\CommentVarTok}[1]{\textcolor[rgb]{0.56,0.35,0.01}{\textbf{\textit{#1}}}}
\newcommand{\ConstantTok}[1]{\textcolor[rgb]{0.00,0.00,0.00}{#1}}
\newcommand{\ControlFlowTok}[1]{\textcolor[rgb]{0.13,0.29,0.53}{\textbf{#1}}}
\newcommand{\DataTypeTok}[1]{\textcolor[rgb]{0.13,0.29,0.53}{#1}}
\newcommand{\DecValTok}[1]{\textcolor[rgb]{0.00,0.00,0.81}{#1}}
\newcommand{\DocumentationTok}[1]{\textcolor[rgb]{0.56,0.35,0.01}{\textbf{\textit{#1}}}}
\newcommand{\ErrorTok}[1]{\textcolor[rgb]{0.64,0.00,0.00}{\textbf{#1}}}
\newcommand{\ExtensionTok}[1]{#1}
\newcommand{\FloatTok}[1]{\textcolor[rgb]{0.00,0.00,0.81}{#1}}
\newcommand{\FunctionTok}[1]{\textcolor[rgb]{0.00,0.00,0.00}{#1}}
\newcommand{\ImportTok}[1]{#1}
\newcommand{\InformationTok}[1]{\textcolor[rgb]{0.56,0.35,0.01}{\textbf{\textit{#1}}}}
\newcommand{\KeywordTok}[1]{\textcolor[rgb]{0.13,0.29,0.53}{\textbf{#1}}}
\newcommand{\NormalTok}[1]{#1}
\newcommand{\OperatorTok}[1]{\textcolor[rgb]{0.81,0.36,0.00}{\textbf{#1}}}
\newcommand{\OtherTok}[1]{\textcolor[rgb]{0.56,0.35,0.01}{#1}}
\newcommand{\PreprocessorTok}[1]{\textcolor[rgb]{0.56,0.35,0.01}{\textit{#1}}}
\newcommand{\RegionMarkerTok}[1]{#1}
\newcommand{\SpecialCharTok}[1]{\textcolor[rgb]{0.00,0.00,0.00}{#1}}
\newcommand{\SpecialStringTok}[1]{\textcolor[rgb]{0.31,0.60,0.02}{#1}}
\newcommand{\StringTok}[1]{\textcolor[rgb]{0.31,0.60,0.02}{#1}}
\newcommand{\VariableTok}[1]{\textcolor[rgb]{0.00,0.00,0.00}{#1}}
\newcommand{\VerbatimStringTok}[1]{\textcolor[rgb]{0.31,0.60,0.02}{#1}}
\newcommand{\WarningTok}[1]{\textcolor[rgb]{0.56,0.35,0.01}{\textbf{\textit{#1}}}}
\usepackage{graphicx,grffile}
\makeatletter
\def\maxwidth{\ifdim\Gin@nat@width>\linewidth\linewidth\else\Gin@nat@width\fi}
\def\maxheight{\ifdim\Gin@nat@height>\textheight\textheight\else\Gin@nat@height\fi}
\makeatother
% Scale images if necessary, so that they will not overflow the page
% margins by default, and it is still possible to overwrite the defaults
% using explicit options in \includegraphics[width, height, ...]{}
\setkeys{Gin}{width=\maxwidth,height=\maxheight,keepaspectratio}
\setlength{\emergencystretch}{3em}  % prevent overfull lines
\providecommand{\tightlist}{%
  \setlength{\itemsep}{0pt}\setlength{\parskip}{0pt}}
\setcounter{secnumdepth}{0}
% Redefines (sub)paragraphs to behave more like sections
\ifx\paragraph\undefined\else
\let\oldparagraph\paragraph
\renewcommand{\paragraph}[1]{\oldparagraph{#1}\mbox{}}
\fi
\ifx\subparagraph\undefined\else
\let\oldsubparagraph\subparagraph
\renewcommand{\subparagraph}[1]{\oldsubparagraph{#1}\mbox{}}
\fi

% set default figure placement to htbp
\makeatletter
\def\fps@figure{htbp}
\makeatother


\title{Assignment 10: Data Scraping}
\author{Rachel Gonsenhauser}
\date{}

\begin{document}
\maketitle

\hypertarget{total-points}{%
\section{Total points:}\label{total-points}}

\hypertarget{overview}{%
\subsection{OVERVIEW}\label{overview}}

This exercise accompanies the lessons in Environmental Data Analytics on
time series analysis.

\hypertarget{directions}{%
\subsection{Directions}\label{directions}}

\begin{enumerate}
\def\labelenumi{\arabic{enumi}.}
\tightlist
\item
  Change ``Student Name'' on line 3 (above) with your name.
\item
  Work through the steps, \textbf{creating code and output} that fulfill
  each instruction.
\item
  Be sure to \textbf{answer the questions} in this assignment document.
\item
  When you have completed the assignment, \textbf{Knit} the text and
  code into a single PDF file.
\item
  After Knitting, submit the completed exercise (PDF file) to the
  dropbox in Sakai. Add your last name into the file name (e.g.,
  ``Salk\_A06\_GLMs\_Week1.Rmd'') prior to submission.
\end{enumerate}

The completed exercise is due on Tuesday, April 7 at 1:00 pm.

\hypertarget{set-up}{%
\subsection{Set up}\label{set-up}}

\begin{enumerate}
\def\labelenumi{\arabic{enumi}.}
\tightlist
\item
  Set up your session:
\end{enumerate}

\begin{itemize}
\tightlist
\item
  Check your working directory
\item
  Load the packages \texttt{tidyverse}, \texttt{rvest}, and any others
  you end up using.
\item
  Set your ggplot theme
\end{itemize}

\begin{Shaded}
\begin{Highlighting}[]
\CommentTok{#getwd()}
\KeywordTok{library}\NormalTok{(tidyverse)}
\KeywordTok{library}\NormalTok{(viridis)}
\CommentTok{#install.packages("rvest")}
\KeywordTok{library}\NormalTok{(rvest)}
\CommentTok{#install.packages("ggrepel")}
\KeywordTok{library}\NormalTok{(ggrepel)}

\NormalTok{mytheme <-}\StringTok{ }\KeywordTok{theme_classic}\NormalTok{() }\OperatorTok{+}
\StringTok{  }\KeywordTok{theme}\NormalTok{(}\DataTypeTok{axis.text =} \KeywordTok{element_text}\NormalTok{(}\DataTypeTok{color =} \StringTok{"black"}\NormalTok{), }
        \DataTypeTok{legend.position =} \StringTok{"top"}\NormalTok{)}
\KeywordTok{theme_set}\NormalTok{(mytheme)}
\end{Highlighting}
\end{Shaded}

\begin{enumerate}
\def\labelenumi{\arabic{enumi}.}
\setcounter{enumi}{1}
\tightlist
\item
  Indicate the EPA impaired waters website
  (\url{https://www.epa.gov/nutrient-policy-data/waters-assessed-impaired-due-nutrient-related-causes})
  as the URL to be scraped.
\end{enumerate}

\begin{Shaded}
\begin{Highlighting}[]
\NormalTok{url <-}\StringTok{ "https://www.epa.gov/nutrient-policy-data/waters-assessed-impaired-due-nutrient-related-causes"}
\NormalTok{webpage <-}\StringTok{ }\KeywordTok{read_html}\NormalTok{(url)}
\end{Highlighting}
\end{Shaded}

\begin{enumerate}
\def\labelenumi{\arabic{enumi}.}
\setcounter{enumi}{2}
\tightlist
\item
  Scrape the Rivers table, with every column except year. Then, turn it
  into a data frame.
\end{enumerate}

\begin{Shaded}
\begin{Highlighting}[]
\NormalTok{State <-}\StringTok{ }\NormalTok{webpage }\OperatorTok\StringTok{ }\KeywordTok{html_nodes}\NormalTok{(}\StringTok{"table:nth-child(8) td:nth-child(1)"}\NormalTok{) }\OperatorTok\StringTok{ }\KeywordTok{html_text}\NormalTok{()}
\NormalTok{Rivers.Assessed.mi2 <-}\StringTok{ }\NormalTok{webpage }\OperatorTok\StringTok{ }\KeywordTok{html_nodes}\NormalTok{(}\StringTok{"table:nth-child(8) td:nth-child(2)"}\NormalTok{) }\OperatorTok\StringTok{ }\KeywordTok{html_text}\NormalTok{()}
\NormalTok{Rivers.Assessed.percent <-}\StringTok{ }\NormalTok{webpage }\OperatorTok\StringTok{ }\KeywordTok{html_nodes}\NormalTok{(}\StringTok{"table:nth-child(8) td:nth-child(3)"}\NormalTok{) }\OperatorTok\StringTok{ }\KeywordTok{html_text}\NormalTok{()}
\NormalTok{Rivers.Impaired.mi2 <-}\StringTok{ }\NormalTok{webpage }\OperatorTok\StringTok{ }\KeywordTok{html_nodes}\NormalTok{(}\StringTok{"table:nth-child(8) td:nth-child(4)"}\NormalTok{) }\OperatorTok\StringTok{ }\KeywordTok{html_text}\NormalTok{()}
\NormalTok{Rivers.Impaired.percent <-}\StringTok{ }\NormalTok{webpage }\OperatorTok\StringTok{ }\KeywordTok{html_nodes}\NormalTok{(}\StringTok{"table:nth-child(8) td:nth-child(5)"}\NormalTok{) }\OperatorTok\StringTok{ }\KeywordTok{html_text}\NormalTok{()}
\NormalTok{Rivers.Impaired.percent.TMDL <-}\StringTok{ }\NormalTok{webpage }\OperatorTok\StringTok{ }\KeywordTok{html_nodes}\NormalTok{(}\StringTok{"table:nth-child(8) td:nth-child(6)"}\NormalTok{) }\OperatorTok\StringTok{ }\KeywordTok{html_text}\NormalTok{()}

\NormalTok{Rivers <-}\StringTok{ }\KeywordTok{data.frame}\NormalTok{(State, Rivers.Assessed.mi2, Rivers.Assessed.percent, }
\NormalTok{                          Rivers.Impaired.mi2, Rivers.Impaired.percent, }
\NormalTok{                          Rivers.Impaired.percent.TMDL)}
\end{Highlighting}
\end{Shaded}

\begin{enumerate}
\def\labelenumi{\arabic{enumi}.}
\setcounter{enumi}{3}
\item
  Use \texttt{str\_replace} to remove non-numeric characters from the
  numeric columns.
\item
  Set the numeric columns to a numeric class and verify this using
  \texttt{str}.
\end{enumerate}

\begin{Shaded}
\begin{Highlighting}[]
\CommentTok{# 4}
\NormalTok{Rivers}\OperatorTok{$}\NormalTok{Rivers.Assessed.mi2 <-}\StringTok{ }\KeywordTok{str_replace}\NormalTok{(Rivers}\OperatorTok{$}\NormalTok{Rivers.Assessed.mi2,}
                                                      \DataTypeTok{pattern =} \StringTok{"([,])"}\NormalTok{, }\DataTypeTok{replacement =} \StringTok{""}\NormalTok{)  }
\NormalTok{Rivers}\OperatorTok{$}\NormalTok{Rivers.Assessed.percent <-}\StringTok{ }\KeywordTok{str_replace}\NormalTok{(Rivers}\OperatorTok{$}\NormalTok{Rivers.Assessed.percent,}
                                                      \DataTypeTok{pattern =} \StringTok{"([%])"}\NormalTok{, }\DataTypeTok{replacement =} \StringTok{""}\NormalTok{)}
\NormalTok{Rivers}\OperatorTok{$}\NormalTok{Rivers.Assessed.percent <-}\StringTok{ }\KeywordTok{str_replace}\NormalTok{(Rivers}\OperatorTok{$}\NormalTok{Rivers.Assessed.percent,}
                                                      \DataTypeTok{pattern =} \StringTok{"([*])"}\NormalTok{, }\DataTypeTok{replacement =} \StringTok{""}\NormalTok{)}
\NormalTok{Rivers}\OperatorTok{$}\NormalTok{Rivers.Impaired.mi2 <-}\StringTok{ }\KeywordTok{str_replace}\NormalTok{(Rivers}\OperatorTok{$}\NormalTok{Rivers.Impaired.mi2,}
                                                      \DataTypeTok{pattern =} \StringTok{"([,])"}\NormalTok{, }\DataTypeTok{replacement =} \StringTok{""}\NormalTok{)  }
\NormalTok{Rivers}\OperatorTok{$}\NormalTok{Rivers.Impaired.percent <-}\StringTok{ }\KeywordTok{str_replace}\NormalTok{(Rivers}\OperatorTok{$}\NormalTok{Rivers.Impaired.percent, }
                                                  \DataTypeTok{pattern =} \StringTok{"([%])"}\NormalTok{, }\DataTypeTok{replacement =} \StringTok{""}\NormalTok{)}
\NormalTok{Rivers}\OperatorTok{$}\NormalTok{Rivers.Impaired.percent.TMDL <-}\StringTok{ }\KeywordTok{str_replace}\NormalTok{(Rivers}\OperatorTok{$}\NormalTok{Rivers.Impaired.percent.TMDL, }
                                                       \DataTypeTok{pattern =} \StringTok{"([%])"}\NormalTok{, }\DataTypeTok{replacement =} \StringTok{""}\NormalTok{)}
\NormalTok{Rivers}\OperatorTok{$}\NormalTok{Rivers.Impaired.percent.TMDL <-}\StringTok{ }\KeywordTok{str_replace}\NormalTok{(Rivers}\OperatorTok{$}\NormalTok{Rivers.Impaired.percent.TMDL, }
                                                       \DataTypeTok{pattern =} \StringTok{"([±])"}\NormalTok{, }\DataTypeTok{replacement =} \StringTok{""}\NormalTok{)}

\CommentTok{# 5}
\KeywordTok{str}\NormalTok{(Rivers)}
\end{Highlighting}
\end{Shaded}

\begin{verbatim}
## 'data.frame':    50 obs. of  6 variables:
##  $ State                       : Factor w/ 50 levels "Alabama","Alaska",..: 1 2 3 4 5 6 7 8 9 10 ...
##  $ Rivers.Assessed.mi2         : chr  "10538" "602" "2764" "9979" ...
##  $ Rivers.Assessed.percent     : chr  "14" "0" "3" "11" ...
##  $ Rivers.Impaired.mi2         : chr  "1146" "15" "144" "1440" ...
##  $ Rivers.Impaired.percent     : chr  "11" "2" "5" "14" ...
##  $ Rivers.Impaired.percent.TMDL: chr  "53" "100" "6" "2" ...
\end{verbatim}

\begin{Shaded}
\begin{Highlighting}[]
\NormalTok{Rivers}\OperatorTok{$}\NormalTok{Rivers.Assessed.mi2 <-}\StringTok{ }\KeywordTok{as.numeric}\NormalTok{(Rivers}\OperatorTok{$}\NormalTok{Rivers.Assessed.mi2)}
\NormalTok{Rivers}\OperatorTok{$}\NormalTok{Rivers.Assessed.percent <-}\StringTok{ }\KeywordTok{as.numeric}\NormalTok{(Rivers}\OperatorTok{$}\NormalTok{Rivers.Assessed.percent)}
\NormalTok{Rivers}\OperatorTok{$}\NormalTok{Rivers.Impaired.mi2 <-}\StringTok{ }\KeywordTok{as.numeric}\NormalTok{(Rivers}\OperatorTok{$}\NormalTok{Rivers.Impaired.mi2)}
\NormalTok{Rivers}\OperatorTok{$}\NormalTok{Rivers.Impaired.percent <-}\StringTok{ }\KeywordTok{as.numeric}\NormalTok{(Rivers}\OperatorTok{$}\NormalTok{Rivers.Impaired.percent)}
\NormalTok{Rivers}\OperatorTok{$}\NormalTok{Rivers.Impaired.percent.TMDL <-}\StringTok{ }\KeywordTok{as.numeric}\NormalTok{(Rivers}\OperatorTok{$}\NormalTok{Rivers.Impaired.percent.TMDL)}
\KeywordTok{str}\NormalTok{(Rivers)}
\end{Highlighting}
\end{Shaded}

\begin{verbatim}
## 'data.frame':    50 obs. of  6 variables:
##  $ State                       : Factor w/ 50 levels "Alabama","Alaska",..: 1 2 3 4 5 6 7 8 9 10 ...
##  $ Rivers.Assessed.mi2         : num  10538 602 2764 9979 32803 ...
##  $ Rivers.Assessed.percent     : num  14 0 3 11 16 56 41 100 20 19 ...
##  $ Rivers.Impaired.mi2         : num  1146 15 144 1440 13350 ...
##  $ Rivers.Impaired.percent     : num  11 2 5 14 41 0 0 88 53 9 ...
##  $ Rivers.Impaired.percent.TMDL: num  53 100 6 2 NA 14 73 37 NA 78 ...
\end{verbatim}

\begin{enumerate}
\def\labelenumi{\arabic{enumi}.}
\setcounter{enumi}{5}
\tightlist
\item
  Scrape the Lakes table, with every column except year. Then, turn it
  into a data frame.
\end{enumerate}

\begin{Shaded}
\begin{Highlighting}[]
\NormalTok{State <-}\StringTok{ }\NormalTok{webpage }\OperatorTok\StringTok{ }\KeywordTok{html_nodes}\NormalTok{(}\StringTok{"table:nth-child(14) td:nth-child(1)"}\NormalTok{) }\OperatorTok\StringTok{ }\KeywordTok{html_text}\NormalTok{()}
\NormalTok{Lakes.Assessed.mi2 <-}\StringTok{ }\NormalTok{webpage }\OperatorTok\StringTok{ }\KeywordTok{html_nodes}\NormalTok{(}\StringTok{"table:nth-child(14) td:nth-child(2)"}\NormalTok{) }\OperatorTok\StringTok{ }\KeywordTok{html_text}\NormalTok{()}
\NormalTok{Lakes.Assessed.percent <-}\StringTok{ }\NormalTok{webpage }\OperatorTok\StringTok{ }\KeywordTok{html_nodes}\NormalTok{(}\StringTok{"table:nth-child(14) td:nth-child(3)"}\NormalTok{) }\OperatorTok\StringTok{ }\KeywordTok{html_text}\NormalTok{()}
\NormalTok{Lakes.Impaired.mi2 <-}\StringTok{ }\NormalTok{webpage }\OperatorTok\StringTok{ }\KeywordTok{html_nodes}\NormalTok{(}\StringTok{"table:nth-child(14) td:nth-child(4)"}\NormalTok{) }\OperatorTok\StringTok{ }\KeywordTok{html_text}\NormalTok{()}
\NormalTok{Lakes.Impaired.percent <-}\StringTok{ }\NormalTok{webpage }\OperatorTok\StringTok{ }\KeywordTok{html_nodes}\NormalTok{(}\StringTok{"table:nth-child(14) td:nth-child(5)"}\NormalTok{) }\OperatorTok\StringTok{ }\KeywordTok{html_text}\NormalTok{()}
\NormalTok{Lakes.Impaired.percent.TMDL <-}\StringTok{ }\NormalTok{webpage }\OperatorTok\StringTok{ }\KeywordTok{html_nodes}\NormalTok{(}\StringTok{"table:nth-child(14) td:nth-child(6)"}\NormalTok{) }\OperatorTok\StringTok{ }\KeywordTok{html_text}\NormalTok{()}

\NormalTok{Lakes <-}\StringTok{ }\KeywordTok{data.frame}\NormalTok{(State, Lakes.Assessed.mi2, Lakes.Assessed.percent, }
\NormalTok{                          Lakes.Impaired.mi2, Lakes.Impaired.percent, }
\NormalTok{                          Lakes.Impaired.percent.TMDL)}
\end{Highlighting}
\end{Shaded}

\begin{enumerate}
\def\labelenumi{\arabic{enumi}.}
\setcounter{enumi}{6}
\item
  Filter out the states with no data.
\item
  Use \texttt{str\_replace} to remove non-numeric characters from the
  numeric columns.
\item
  Set the numeric columns to a numeric class and verify this using
  \texttt{str}.
\end{enumerate}

\begin{Shaded}
\begin{Highlighting}[]
\CommentTok{# 7}
\NormalTok{Lakes <-}\StringTok{ }\NormalTok{Lakes }\OperatorTok
\StringTok{  }\KeywordTok{filter}\NormalTok{(State }\OperatorTok{!=}\StringTok{ "Hawaii"} \OperatorTok{&}\StringTok{ }\NormalTok{State }\OperatorTok{!=}\StringTok{ "Pennsylvania"}\NormalTok{)}

\CommentTok{# 8}
\NormalTok{Lakes}\OperatorTok{$}\NormalTok{Lakes.Assessed.mi2 <-}\StringTok{ }\KeywordTok{str_replace}\NormalTok{(Lakes}\OperatorTok{$}\NormalTok{Lakes.Assessed.mi2,}
                                                      \DataTypeTok{pattern =} \StringTok{"([,])"}\NormalTok{, }\DataTypeTok{replacement =} \StringTok{""}\NormalTok{)  }
\NormalTok{Lakes}\OperatorTok{$}\NormalTok{Lakes.Assessed.percent <-}\StringTok{ }\KeywordTok{str_replace}\NormalTok{(Lakes}\OperatorTok{$}\NormalTok{Lakes.Assessed.percent,}
                                                      \DataTypeTok{pattern =} \StringTok{"([%])"}\NormalTok{, }\DataTypeTok{replacement =} \StringTok{""}\NormalTok{)}
\NormalTok{Lakes}\OperatorTok{$}\NormalTok{Lakes.Assessed.percent <-}\StringTok{ }\KeywordTok{str_replace}\NormalTok{(Lakes}\OperatorTok{$}\NormalTok{Lakes.Assessed.percent,}
                                                      \DataTypeTok{pattern =} \StringTok{"([*])"}\NormalTok{, }\DataTypeTok{replacement =} \StringTok{""}\NormalTok{)}
\NormalTok{Lakes}\OperatorTok{$}\NormalTok{Lakes.Impaired.mi2 <-}\StringTok{ }\KeywordTok{str_replace}\NormalTok{(Lakes}\OperatorTok{$}\NormalTok{Lakes.Impaired.mi2,}
                                                      \DataTypeTok{pattern =} \StringTok{"([,])"}\NormalTok{, }\DataTypeTok{replacement =} \StringTok{""}\NormalTok{)  }
\NormalTok{Lakes}\OperatorTok{$}\NormalTok{Lakes.Impaired.percent <-}\StringTok{ }\KeywordTok{str_replace}\NormalTok{(Lakes}\OperatorTok{$}\NormalTok{Lakes.Impaired.percent, }
                                                  \DataTypeTok{pattern =} \StringTok{"([%])"}\NormalTok{, }\DataTypeTok{replacement =} \StringTok{""}\NormalTok{)}
\NormalTok{Lakes}\OperatorTok{$}\NormalTok{Lakes.Impaired.percent.TMDL <-}\StringTok{ }\KeywordTok{str_replace}\NormalTok{(Lakes}\OperatorTok{$}\NormalTok{Lakes.Impaired.percent.TMDL, }
                                                       \DataTypeTok{pattern =} \StringTok{"([%])"}\NormalTok{, }\DataTypeTok{replacement =} \StringTok{""}\NormalTok{)}
\NormalTok{Lakes}\OperatorTok{$}\NormalTok{Lakes.Impaired.percent.TMDL <-}\StringTok{ }\KeywordTok{str_replace}\NormalTok{(Lakes}\OperatorTok{$}\NormalTok{Lakes.Impaired.percent.TMDL, }
                                                       \DataTypeTok{pattern =} \StringTok{"([±])"}\NormalTok{, }\DataTypeTok{replacement =} \StringTok{""}\NormalTok{)}

\CommentTok{# 9}
\KeywordTok{str}\NormalTok{(Lakes)}
\end{Highlighting}
\end{Shaded}

\begin{verbatim}
## 'data.frame':    48 obs. of  6 variables:
##  $ State                      : Factor w/ 50 levels "Alabama","Alaska",..: 1 2 3 4 5 6 7 8 9 10 ...
##  $ Lakes.Assessed.mi2         : chr  "430.976" "5981" "114976" "64778" ...
##  $ Lakes.Assessed.percent     : chr  "88" "0" "34" "13" ...
##  $ Lakes.Impaired.mi2         : chr  "81740" "1137" "4895" "6513" ...
##  $ Lakes.Impaired.percent     : chr  "19" "19" "4" "10" ...
##  $ Lakes.Impaired.percent.TMDL: chr  "53" "73" "9" "71" ...
\end{verbatim}

\begin{Shaded}
\begin{Highlighting}[]
\NormalTok{Lakes}\OperatorTok{$}\NormalTok{Lakes.Assessed.mi2 <-}\StringTok{ }\KeywordTok{as.numeric}\NormalTok{(Lakes}\OperatorTok{$}\NormalTok{Lakes.Assessed.mi2)}
\end{Highlighting}
\end{Shaded}

\begin{verbatim}
## Warning: NAs introduced by coercion
\end{verbatim}

\begin{Shaded}
\begin{Highlighting}[]
\NormalTok{Lakes}\OperatorTok{$}\NormalTok{Lakes.Assessed.percent <-}\StringTok{ }\KeywordTok{as.numeric}\NormalTok{(Lakes}\OperatorTok{$}\NormalTok{Lakes.Assessed.percent)}
\NormalTok{Lakes}\OperatorTok{$}\NormalTok{Lakes.Impaired.mi2 <-}\StringTok{ }\KeywordTok{as.numeric}\NormalTok{(Lakes}\OperatorTok{$}\NormalTok{Lakes.Impaired.mi2)}
\NormalTok{Lakes}\OperatorTok{$}\NormalTok{Lakes.Impaired.percent <-}\StringTok{ }\KeywordTok{as.numeric}\NormalTok{(Lakes}\OperatorTok{$}\NormalTok{Lakes.Impaired.percent)}
\NormalTok{Lakes}\OperatorTok{$}\NormalTok{Lakes.Impaired.percent.TMDL <-}\StringTok{ }\KeywordTok{as.numeric}\NormalTok{(Lakes}\OperatorTok{$}\NormalTok{Lakes.Impaired.percent.TMDL)}
\KeywordTok{str}\NormalTok{(Lakes)}
\end{Highlighting}
\end{Shaded}

\begin{verbatim}
## 'data.frame':    48 obs. of  6 variables:
##  $ State                      : Factor w/ 50 levels "Alabama","Alaska",..: 1 2 3 4 5 6 7 8 9 10 ...
##  $ Lakes.Assessed.mi2         : num  431 5981 114976 64778 NA ...
##  $ Lakes.Assessed.percent     : num  88 0 34 13 50 95 47 100 54 82 ...
##  $ Lakes.Impaired.mi2         : num  81740 1137 4895 6513 473954 ...
##  $ Lakes.Impaired.percent     : num  19 19 4 10 45 7 12 88 82 2 ...
##  $ Lakes.Impaired.percent.TMDL: num  53 73 9 71 NA 0 7 69 NA 20 ...
\end{verbatim}

\begin{enumerate}
\def\labelenumi{\arabic{enumi}.}
\setcounter{enumi}{9}
\tightlist
\item
  Join the two data frames with a \texttt{full\_join}.
\end{enumerate}

\begin{Shaded}
\begin{Highlighting}[]
\NormalTok{Rivers.and.Lakes <-}\StringTok{ }\KeywordTok{full_join}\NormalTok{(Rivers, Lakes)}
\end{Highlighting}
\end{Shaded}

\begin{verbatim}
## Joining, by = "State"
\end{verbatim}

\begin{enumerate}
\def\labelenumi{\arabic{enumi}.}
\setcounter{enumi}{10}
\tightlist
\item
  Create one graph that compares the data for lakes and/or rivers. This
  option is flexible; choose a relationship (or relationships) that seem
  interesting to you, and think about the implications of your findings.
  This graph should be edited so it follows best data visualization
  practices.
\end{enumerate}

(You may choose to run a statistical test or add a line of best fit;
this is optional but may aid in your interpretations)

\begin{Shaded}
\begin{Highlighting}[]
\CommentTok{# Rivers graph (Figure 1)}
\KeywordTok{ggplot}\NormalTok{(Rivers.and.Lakes, }\KeywordTok{aes}\NormalTok{(}\DataTypeTok{x =}\NormalTok{ Rivers.Assessed.mi2, }\DataTypeTok{y =}\NormalTok{ Rivers.Impaired.mi2, }
                             \DataTypeTok{fill =}\NormalTok{ Rivers.Impaired.percent.TMDL)) }\OperatorTok{+}
\StringTok{  }\KeywordTok{geom_point}\NormalTok{(}\DataTypeTok{shape =} \DecValTok{21}\NormalTok{, }\DataTypeTok{size =} \DecValTok{2}\NormalTok{, }\DataTypeTok{alpha =} \FloatTok{0.8}\NormalTok{) }\OperatorTok{+}\StringTok{ }
\StringTok{  }\KeywordTok{scale_fill_viridis_c}\NormalTok{(}\DataTypeTok{option =} \StringTok{"inferno"}\NormalTok{, }\DataTypeTok{begin =} \FloatTok{0.2}\NormalTok{, }\DataTypeTok{end =} \FloatTok{0.9}\NormalTok{, }\DataTypeTok{direction =} \DecValTok{-1}\NormalTok{) }\OperatorTok{+}
\StringTok{  }\KeywordTok{geom_label_repel}\NormalTok{(}\DataTypeTok{data =} \KeywordTok{subset}\NormalTok{(Rivers.and.Lakes , State }\OperatorTok\StringTok{ }\KeywordTok{c}\NormalTok{(}\StringTok{"Ohio"}\NormalTok{, }\StringTok{"Oregon"}\NormalTok{, }\StringTok{"California"}\NormalTok{, }\StringTok{"Idaho"}\NormalTok{, }\StringTok{"Kansas"}\NormalTok{, }\StringTok{"Michigan"}\NormalTok{, }\StringTok{"Pennsylvania"}\NormalTok{, }\StringTok{"Maine"}\NormalTok{, }\StringTok{"North Dakota"}\NormalTok{, }\StringTok{"Colorado"}\NormalTok{)), }
  \KeywordTok{aes}\NormalTok{(}\DataTypeTok{label =}\NormalTok{ State), }\DataTypeTok{nudge_x =} \DecValTok{-500}\NormalTok{, }\DataTypeTok{nudge_y =} \DecValTok{200}\NormalTok{) }\OperatorTok{+}
\StringTok{  }\KeywordTok{labs}\NormalTok{(}\DataTypeTok{x =} \KeywordTok{expression}\NormalTok{(}\StringTok{"River Area Assessed (mi"}\OperatorTok{^}\StringTok{"2"}\OperatorTok{*}\StringTok{")"}\NormalTok{),}
   \DataTypeTok{y =} \KeywordTok{expression}\NormalTok{(}\StringTok{"River Area Impaired (mi"}\OperatorTok{^}\StringTok{"2"}\OperatorTok{*}\StringTok{")"}\NormalTok{), }
  \DataTypeTok{fill =} \StringTok{"% Impaired with TMDL"}\NormalTok{)}
\end{Highlighting}
\end{Shaded}

\includegraphics{A10_DataScraping_files/figure-latex/unnamed-chunk-8-1.pdf}

\begin{Shaded}
\begin{Highlighting}[]
\CommentTok{# Rivers simple linear regression}
\NormalTok{Rivers.regression <-}\StringTok{ }\KeywordTok{lm}\NormalTok{(}\DataTypeTok{data =}\NormalTok{ Rivers.and.Lakes, Rivers.Impaired.mi2 }\OperatorTok{~}\StringTok{ }\NormalTok{Rivers.Assessed.mi2 }\OperatorTok{+}\StringTok{ }\NormalTok{Rivers.Impaired.percent.TMDL)}
\KeywordTok{summary}\NormalTok{(Rivers.regression)}
\end{Highlighting}
\end{Shaded}

\begin{verbatim}
## 
## Call:
## lm(formula = Rivers.Impaired.mi2 ~ Rivers.Assessed.mi2 + Rivers.Impaired.percent.TMDL, 
##     data = Rivers.and.Lakes)
## 
## Residuals:
##     Min      1Q  Median      3Q     Max 
## -2585.2 -1191.8  -422.8   215.4  6025.8 
## 
## Coefficients:
##                                Estimate Std. Error t value Pr(>|t|)  
## (Intercept)                  1389.78579  525.77907   2.643   0.0121 *
## Rivers.Assessed.mi2             0.02586    0.01529   1.691   0.0994 .
## Rivers.Impaired.percent.TMDL   -4.69516   12.11993  -0.387   0.7007  
## ---
## Signif. codes:  0 '***' 0.001 '**' 0.01 '*' 0.05 '.' 0.1 ' ' 1
## 
## Residual standard error: 2048 on 36 degrees of freedom
##   (11 observations deleted due to missingness)
## Multiple R-squared:  0.08217,    Adjusted R-squared:  0.03118 
## F-statistic: 1.611 on 2 and 36 DF,  p-value: 0.2137
\end{verbatim}

\begin{Shaded}
\begin{Highlighting}[]
\CommentTok{# Lakes graph (Figure 2)}
\KeywordTok{ggplot}\NormalTok{(Rivers.and.Lakes, }\KeywordTok{aes}\NormalTok{(}\DataTypeTok{x =}\NormalTok{ Lakes.Assessed.mi2, }\DataTypeTok{y =}\NormalTok{ Lakes.Impaired.mi2,}
                             \DataTypeTok{fill =}\NormalTok{ Lakes.Impaired.percent.TMDL)) }\OperatorTok{+}
\StringTok{  }\KeywordTok{geom_point}\NormalTok{(}\DataTypeTok{shape =} \DecValTok{21}\NormalTok{, }\DataTypeTok{size =} \DecValTok{2}\NormalTok{, }\DataTypeTok{alpha =} \FloatTok{0.8}\NormalTok{) }\OperatorTok{+}\StringTok{ }
\StringTok{  }\KeywordTok{scale_fill_viridis_c}\NormalTok{(}\DataTypeTok{option =} \StringTok{"inferno"}\NormalTok{, }\DataTypeTok{begin =} \FloatTok{0.2}\NormalTok{, }\DataTypeTok{end =} \FloatTok{0.9}\NormalTok{, }\DataTypeTok{direction =} \DecValTok{-1}\NormalTok{) }\OperatorTok{+}
\StringTok{  }\KeywordTok{geom_label_repel}\NormalTok{(}\DataTypeTok{data =} \KeywordTok{subset}\NormalTok{(Rivers.and.Lakes, State }\OperatorTok\StringTok{ }\KeywordTok{c}\NormalTok{(}\StringTok{"Oklahoma"}\NormalTok{, }\StringTok{"Wisconsin"}\NormalTok{, }\StringTok{"North Dakota"}\NormalTok{, }\StringTok{"Michigan"}\NormalTok{, }\StringTok{"Louisiana"}\NormalTok{, }\StringTok{"Utah"}\NormalTok{, }\StringTok{"New York"}\NormalTok{, }\StringTok{"Montana"}\NormalTok{, }\StringTok{"Washington"}\NormalTok{, }\StringTok{"Tennessee"}\NormalTok{)), }
  \KeywordTok{aes}\NormalTok{(}\DataTypeTok{label =}\NormalTok{ State), }\DataTypeTok{nudge_x =} \DecValTok{-500}\NormalTok{, }\DataTypeTok{nudge_y =} \DecValTok{200}\NormalTok{) }\OperatorTok{+}
\StringTok{  }\KeywordTok{labs}\NormalTok{(}\DataTypeTok{x =} \KeywordTok{expression}\NormalTok{(}\StringTok{"Lake Area Assessed (mi"}\OperatorTok{^}\StringTok{"2"}\OperatorTok{*}\StringTok{")"}\NormalTok{),}
   \DataTypeTok{y =} \KeywordTok{expression}\NormalTok{(}\StringTok{"Lake Area Impaired (mi"}\OperatorTok{^}\StringTok{"2"}\OperatorTok{*}\StringTok{")"}\NormalTok{), }
  \DataTypeTok{fill =} \StringTok{"% Impaired with TMDL"}\NormalTok{)}
\end{Highlighting}
\end{Shaded}

\begin{verbatim}
## Warning: Removed 7 rows containing missing values (geom_point).
\end{verbatim}

\includegraphics{A10_DataScraping_files/figure-latex/unnamed-chunk-8-2.pdf}

\begin{Shaded}
\begin{Highlighting}[]
\CommentTok{# Lakes simple linear regression}
\NormalTok{Lakes.regression <-}\StringTok{ }\KeywordTok{lm}\NormalTok{(}\DataTypeTok{data =}\NormalTok{ Rivers.and.Lakes, Lakes.Impaired.mi2 }\OperatorTok{~}\StringTok{ }\NormalTok{Lakes.Assessed.mi2 }\OperatorTok{+}\StringTok{ }\NormalTok{Lakes.Impaired.percent.TMDL)}
\KeywordTok{summary}\NormalTok{(Lakes.regression)}
\end{Highlighting}
\end{Shaded}

\begin{verbatim}
## 
## Call:
## lm(formula = Lakes.Impaired.mi2 ~ Lakes.Assessed.mi2 + Lakes.Impaired.percent.TMDL, 
##     data = Rivers.and.Lakes)
## 
## Residuals:
##     Min      1Q  Median      3Q     Max 
## -145661  -30230  -12077   18536  112022 
## 
## Coefficients:
##                              Estimate Std. Error t value Pr(>|t|)    
## (Intercept)                 8.589e+03  1.708e+04   0.503 0.619071    
## Lakes.Assessed.mi2          1.630e-01  4.388e-02   3.715 0.000898 ***
## Lakes.Impaired.percent.TMDL 3.208e+02  3.558e+02   0.902 0.374866    
## ---
## Signif. codes:  0 '***' 0.001 '**' 0.01 '*' 0.05 '.' 0.1 ' ' 1
## 
## Residual standard error: 57890 on 28 degrees of freedom
##   (19 observations deleted due to missingness)
## Multiple R-squared:  0.3317, Adjusted R-squared:  0.2839 
## F-statistic: 6.948 on 2 and 28 DF,  p-value: 0.003546
\end{verbatim}

\begin{enumerate}
\def\labelenumi{\arabic{enumi}.}
\setcounter{enumi}{11}
\tightlist
\item
  Summarize the findings that accompany your graph. You may choose to
  suggest further research or data collection to help explain the
  results.
\end{enumerate}

\begin{quote}
Neither the area of rivers assessed nor the percent of impaired rivers
that have all impairments addressed by a TMDL/other restoration plan are
significant indicators of the area of rivers impaired (Figure 1;
multiple linear regression: R2=0.03118, df=2 and 36, p=0.2137).
Conversely, while the percent of impaired rivers that have all
impairments addressed by a TMDL/other restoration plan is not a
significant indicator of the area of rivers impaired, the area of rivers
assessed is a significant indicator of the area found to be impaired
(Figure 2; multiple linear regression: R2=0.2839, df=2 and 28,
p=0.003546). Many states were missing observations for the percent of
impaired rivers or lakes whose impairments are addressed by a TMDL/other
restoration plan, so further data collection in these states is
necessary to establish whether a relationship between this variable and
others studied exists.
\end{quote}

\end{document}
