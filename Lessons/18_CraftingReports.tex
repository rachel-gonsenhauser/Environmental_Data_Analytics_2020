\PassOptionsToPackage{unicode=true}{hyperref} % options for packages loaded elsewhere
\PassOptionsToPackage{hyphens}{url}
%
\documentclass[]{article}
\usepackage{lmodern}
\usepackage{amssymb,amsmath}
\usepackage{ifxetex,ifluatex}
\usepackage{fixltx2e} % provides \textsubscript
\ifnum 0\ifxetex 1\fi\ifluatex 1\fi=0 % if pdftex
  \usepackage[T1]{fontenc}
  \usepackage[utf8]{inputenc}
  \usepackage{textcomp} % provides euro and other symbols
\else % if luatex or xelatex
  \usepackage{unicode-math}
  \defaultfontfeatures{Ligatures=TeX,Scale=MatchLowercase}
\fi
% use upquote if available, for straight quotes in verbatim environments
\IfFileExists{upquote.sty}{\usepackage{upquote}}{}
% use microtype if available
\IfFileExists{microtype.sty}{%
\usepackage[]{microtype}
\UseMicrotypeSet[protrusion]{basicmath} % disable protrusion for tt fonts
}{}
\IfFileExists{parskip.sty}{%
\usepackage{parskip}
}{% else
\setlength{\parindent}{0pt}
\setlength{\parskip}{6pt plus 2pt minus 1pt}
}
\usepackage{hyperref}
\hypersetup{
            pdftitle={17: Crafting Reports},
            pdfauthor={Environmental Data Analytics \textbar{} Kateri Salk},
            pdfborder={0 0 0},
            breaklinks=true}
\urlstyle{same}  % don't use monospace font for urls
\usepackage[margin=2.54cm]{geometry}
\usepackage{color}
\usepackage{fancyvrb}
\newcommand{\VerbBar}{|}
\newcommand{\VERB}{\Verb[commandchars=\\\{\}]}
\DefineVerbatimEnvironment{Highlighting}{Verbatim}{commandchars=\\\{\}}
% Add ',fontsize=\small' for more characters per line
\usepackage{framed}
\definecolor{shadecolor}{RGB}{248,248,248}
\newenvironment{Shaded}{\begin{snugshade}}{\end{snugshade}}
\newcommand{\AlertTok}[1]{\textcolor[rgb]{0.94,0.16,0.16}{#1}}
\newcommand{\AnnotationTok}[1]{\textcolor[rgb]{0.56,0.35,0.01}{\textbf{\textit{#1}}}}
\newcommand{\AttributeTok}[1]{\textcolor[rgb]{0.77,0.63,0.00}{#1}}
\newcommand{\BaseNTok}[1]{\textcolor[rgb]{0.00,0.00,0.81}{#1}}
\newcommand{\BuiltInTok}[1]{#1}
\newcommand{\CharTok}[1]{\textcolor[rgb]{0.31,0.60,0.02}{#1}}
\newcommand{\CommentTok}[1]{\textcolor[rgb]{0.56,0.35,0.01}{\textit{#1}}}
\newcommand{\CommentVarTok}[1]{\textcolor[rgb]{0.56,0.35,0.01}{\textbf{\textit{#1}}}}
\newcommand{\ConstantTok}[1]{\textcolor[rgb]{0.00,0.00,0.00}{#1}}
\newcommand{\ControlFlowTok}[1]{\textcolor[rgb]{0.13,0.29,0.53}{\textbf{#1}}}
\newcommand{\DataTypeTok}[1]{\textcolor[rgb]{0.13,0.29,0.53}{#1}}
\newcommand{\DecValTok}[1]{\textcolor[rgb]{0.00,0.00,0.81}{#1}}
\newcommand{\DocumentationTok}[1]{\textcolor[rgb]{0.56,0.35,0.01}{\textbf{\textit{#1}}}}
\newcommand{\ErrorTok}[1]{\textcolor[rgb]{0.64,0.00,0.00}{\textbf{#1}}}
\newcommand{\ExtensionTok}[1]{#1}
\newcommand{\FloatTok}[1]{\textcolor[rgb]{0.00,0.00,0.81}{#1}}
\newcommand{\FunctionTok}[1]{\textcolor[rgb]{0.00,0.00,0.00}{#1}}
\newcommand{\ImportTok}[1]{#1}
\newcommand{\InformationTok}[1]{\textcolor[rgb]{0.56,0.35,0.01}{\textbf{\textit{#1}}}}
\newcommand{\KeywordTok}[1]{\textcolor[rgb]{0.13,0.29,0.53}{\textbf{#1}}}
\newcommand{\NormalTok}[1]{#1}
\newcommand{\OperatorTok}[1]{\textcolor[rgb]{0.81,0.36,0.00}{\textbf{#1}}}
\newcommand{\OtherTok}[1]{\textcolor[rgb]{0.56,0.35,0.01}{#1}}
\newcommand{\PreprocessorTok}[1]{\textcolor[rgb]{0.56,0.35,0.01}{\textit{#1}}}
\newcommand{\RegionMarkerTok}[1]{#1}
\newcommand{\SpecialCharTok}[1]{\textcolor[rgb]{0.00,0.00,0.00}{#1}}
\newcommand{\SpecialStringTok}[1]{\textcolor[rgb]{0.31,0.60,0.02}{#1}}
\newcommand{\StringTok}[1]{\textcolor[rgb]{0.31,0.60,0.02}{#1}}
\newcommand{\VariableTok}[1]{\textcolor[rgb]{0.00,0.00,0.00}{#1}}
\newcommand{\VerbatimStringTok}[1]{\textcolor[rgb]{0.31,0.60,0.02}{#1}}
\newcommand{\WarningTok}[1]{\textcolor[rgb]{0.56,0.35,0.01}{\textbf{\textit{#1}}}}
\usepackage{longtable,booktabs}
% Fix footnotes in tables (requires footnote package)
\IfFileExists{footnote.sty}{\usepackage{footnote}\makesavenoteenv{longtable}}{}
\usepackage{graphicx,grffile}
\makeatletter
\def\maxwidth{\ifdim\Gin@nat@width>\linewidth\linewidth\else\Gin@nat@width\fi}
\def\maxheight{\ifdim\Gin@nat@height>\textheight\textheight\else\Gin@nat@height\fi}
\makeatother
% Scale images if necessary, so that they will not overflow the page
% margins by default, and it is still possible to overwrite the defaults
% using explicit options in \includegraphics[width, height, ...]{}
\setkeys{Gin}{width=\maxwidth,height=\maxheight,keepaspectratio}
\setlength{\emergencystretch}{3em}  % prevent overfull lines
\providecommand{\tightlist}{%
  \setlength{\itemsep}{0pt}\setlength{\parskip}{0pt}}
\setcounter{secnumdepth}{0}
% Redefines (sub)paragraphs to behave more like sections
\ifx\paragraph\undefined\else
\let\oldparagraph\paragraph
\renewcommand{\paragraph}[1]{\oldparagraph{#1}\mbox{}}
\fi
\ifx\subparagraph\undefined\else
\let\oldsubparagraph\subparagraph
\renewcommand{\subparagraph}[1]{\oldsubparagraph{#1}\mbox{}}
\fi

% set default figure placement to htbp
\makeatletter
\def\fps@figure{htbp}
\makeatother


\title{17: Crafting Reports}
\author{Environmental Data Analytics \textbar{} Kateri Salk}
\date{Spring 2019}

\begin{document}
\maketitle

\hypertarget{lesson-objectives}{%
\subsection{LESSON OBJECTIVES}\label{lesson-objectives}}

\begin{enumerate}
\def\labelenumi{\arabic{enumi}.}
\tightlist
\item
  Describe the purpose of using R Markdown as a communication and
  workflow tool
\item
  Incorporate Markdown syntax into documents
\item
  Communicate the process and findings of an analysis session in the
  style of a report
\end{enumerate}

\hypertarget{basic-r-markdown-document-structure}{%
\subsection{BASIC R MARKDOWN DOCUMENT
STRUCTURE}\label{basic-r-markdown-document-structure}}

\begin{enumerate}
\def\labelenumi{\arabic{enumi}.}
\tightlist
\item
  \textbf{YAML Header} surrounded by --- on top and bottom

  \begin{itemize}
  \tightlist
  \item
    YAML templates include options for html, pdf, word, markdown, and
    interactive
  \item
    More information on formatting the YAML header can be found in the
    cheat sheet
  \end{itemize}
\item
  \textbf{R Code Chunks} surrounded by
  ``\texttt{on\ top\ and\ bottom\ \ +\ Create\ using}Cmd/Ctrl\texttt{+}Alt\texttt{+}I`

  \begin{itemize}
  \tightlist
  \item
    Can be named \{r name\} to facilitate navigation and autoreferencing
  \item
    Chunk options allow for flexibility when the code runs and when the
    document is knitted
  \end{itemize}
\end{enumerate}

\begin{Shaded}
\begin{Highlighting}[]
\CommentTok{# Create this R chunk easily by typing: command, option, I }
\end{Highlighting}
\end{Shaded}

\begin{enumerate}
\def\labelenumi{\arabic{enumi}.}
\setcounter{enumi}{2}
\tightlist
\item
  \textbf{Text} with formatting options for readability in knitted
  document
\end{enumerate}

A handy cheat sheet for R markdown can be found
\href{https://www.rstudio.com/wp-content/uploads/2015/03/rmarkdown-reference.pdf}{here}.
Another one can be found
\href{https://www.rstudio.com/wp-content/uploads/2015/02/rmarkdown-cheatsheet.pdf}{here}.

\hypertarget{why-r-markdown}{%
\subsection{WHY R MARKDOWN?}\label{why-r-markdown}}

\textless{}Fill in our discussion below with bullet points. Use italics
and bold for emphasis (hint: use the cheat sheets to figure out how to
make bold and italic text).\textgreater{}

\begin{itemize}
\item
  Note: instructions written with \textless{}\textgreater{} should not
  appear in knitted PDF
\item
  \emph{italic}
\item
  \textbf{bold}
\item
  An R markdown file allows you to incorporate \textbf{figures, tables,
  and text} all in the same file
\item
  Additionally, this format allows for you to \emph{provide comments} on
  your code and output so that collaborators can easily understand what
  you've done and reproduce analyses
\item
  Can easily display the code, output, and accompanying written material
  in a \textbf{PDF format}
\end{itemize}

\hypertarget{text-editing-challenge}{%
\subsection{TEXT EDITING CHALLENGE}\label{text-editing-challenge}}

Create a table below that details the example datasets we have been
using in class. The first column should contain the names of the
datasets and the second column should include some relevant information
about the datasets. (Hint: use the cheat sheets to figure out how to
make a table in Rmd)

\begin{longtable}[]{@{}ll@{}}
\toprule
\begin{minipage}[b]{0.42\columnwidth}\raggedright
Dataset Name\strut
\end{minipage} & \begin{minipage}[b]{0.52\columnwidth}\raggedright
Relevant Dataset Information\strut
\end{minipage}\tabularnewline
\midrule
\endhead
\begin{minipage}[t]{0.42\columnwidth}\raggedright
EPA Air pollutants measurement\strut
\end{minipage} & \begin{minipage}[t]{0.52\columnwidth}\raggedright
Provides measurements in Ozone and PM\strut
\end{minipage}\tabularnewline
\begin{minipage}[t]{0.42\columnwidth}\raggedright
EXOTOX\strut
\end{minipage} & \begin{minipage}[t]{0.52\columnwidth}\raggedright
Provides data from studies on several neonicotinoids and their effects
on mortality of various organisms\strut
\end{minipage}\tabularnewline
\bottomrule
\end{longtable}

\begin{itemize}
\tightlist
\item
  Kateri prefers to use the kable function to make tables, this is
  sometimes glitchy
\end{itemize}

\hypertarget{r-chunk-editing-challenge}{%
\subsection{R CHUNK EDITING CHALLENGE}\label{r-chunk-editing-challenge}}

\hypertarget{installing-packages}{%
\subsubsection{Installing packages}\label{installing-packages}}

Create an R chunk below that installs the package \texttt{knitr}.
Instead of commenting out the code, customize the chunk options such
that the code is not evaluated (i.e., not run).

\hypertarget{setup}{%
\subsubsection{Setup}\label{setup}}

Create an R chunk below called ``setup'' that checks your working
directory, loads the packages \texttt{tidyverse} and \texttt{knitr}, and
sets a ggplot theme. Remember that you need to disable R throwing a
message, which containts a check mark that cannot be knitted.

Load the NTL-LTER\_Lake\_Nutrients\_Raw dataset, display the head of the
dataset, and set the date column to a date format.

Customize the chunk options such that the code is run but is not
displayed in the final document.

\begin{Shaded}
\begin{Highlighting}[]
\KeywordTok{getwd}\NormalTok{()}
\end{Highlighting}
\end{Shaded}

\begin{verbatim}
## [1] "/Users/rachelgonsenhauser/Documents/Environmental_Data_Analytics_2020"
\end{verbatim}

\begin{Shaded}
\begin{Highlighting}[]
\KeywordTok{library}\NormalTok{(tidyverse)}
\KeywordTok{library}\NormalTok{(knitr)}

\NormalTok{mytheme <-}\StringTok{ }\KeywordTok{theme_classic}\NormalTok{() }\OperatorTok{+}
\StringTok{  }\KeywordTok{theme}\NormalTok{(}\DataTypeTok{axis.text =} \KeywordTok{element_text}\NormalTok{(}\DataTypeTok{color =} \StringTok{"black"}\NormalTok{), }
        \DataTypeTok{legend.position =} \StringTok{"top"}\NormalTok{)}
\KeywordTok{theme_set}\NormalTok{(mytheme)}

\NormalTok{Nutrients.raw <-}\StringTok{ }\KeywordTok{read.csv}\NormalTok{(}\StringTok{"./Data/Raw/NTL-LTER_Lake_Nutrients_Raw.csv"}\NormalTok{)}
\KeywordTok{head}\NormalTok{(Nutrients.raw)}
\end{Highlighting}
\end{Shaded}

\begin{verbatim}
##   lakeid  lakename year4 daynum sampledate depth_id depth tn_ug tp_ug nh34 no23
## 1      L Paul Lake  1991    140    5/20/91        1  0.00   538    25   NA   NA
## 2      L Paul Lake  1991    140    5/20/91        2  0.85   285    14   NA   NA
## 3      L Paul Lake  1991    140    5/20/91        3  1.75   399    14   NA   NA
## 4      L Paul Lake  1991    140    5/20/91        4  3.00   453    14   NA   NA
## 5      L Paul Lake  1991    140    5/20/91        5  4.00   363    13   NA   NA
## 6      L Paul Lake  1991    140    5/20/91        6  6.00   583    37   NA   NA
##   po4 comments
## 1  NA         
## 2  NA         
## 3  NA         
## 4  NA         
## 5  NA         
## 6  NA
\end{verbatim}

\begin{Shaded}
\begin{Highlighting}[]
\NormalTok{Nutrients.raw}\OperatorTok{$}\NormalTok{sampledate <-}\StringTok{ }\KeywordTok{as.Date}\NormalTok{(Nutrients.raw}\OperatorTok{$}\NormalTok{sampledate, }\DataTypeTok{format =} \StringTok{"%m/%d/%y"}\NormalTok{)}
\KeywordTok{class}\NormalTok{(Nutrients.raw}\OperatorTok{$}\NormalTok{sampledate)}
\end{Highlighting}
\end{Shaded}

\begin{verbatim}
## [1] "Date"
\end{verbatim}

\hypertarget{data-exploration-wrangling-and-visualization}{%
\subsubsection{Data Exploration, Wrangling, and
Visualization}\label{data-exploration-wrangling-and-visualization}}

Create an R chunk below to create a processed dataset do the following
operations:

\begin{itemize}
\tightlist
\item
  Include all columns except lakeid, depth\_id, and comments
\item
  Include only surface samples (depth = 0 m)
\end{itemize}

\begin{Shaded}
\begin{Highlighting}[]
\NormalTok{Nutrients.processed <-}\StringTok{ }
\StringTok{  }\NormalTok{Nutrients.raw }\OperatorTok
\StringTok{  }\KeywordTok{select}\NormalTok{(lakename}\OperatorTok{:}\NormalTok{sampledate, depth}\OperatorTok{:}\NormalTok{po4) }\OperatorTok
\StringTok{  }\KeywordTok{filter}\NormalTok{(depth }\OperatorTok{==}\StringTok{ }\DecValTok{0}\NormalTok{)}

\CommentTok{#could also code like this to exlucde rows we don't want}
\CommentTok{#Nutrients.processed <- }
  \CommentTok{#Nutrients.raw %>%}
  \CommentTok{#select(-lakeid,-depth_id,-comments) %>%}
  \CommentTok{#filter(depth == 0)}
\end{Highlighting}
\end{Shaded}

Create a second R chunk to create a summary dataset with the mean,
minimum, maximum, and standard deviation of total nitrogen
concentrations for each lake. Create a second summary dataset that is
identical except that it evaluates total phosphorus. Customize the chunk
options such that the code is run but not displayed in the final
document.

\begin{Shaded}
\begin{Highlighting}[]
\NormalTok{Nitrogen.summary <-}\StringTok{ }
\StringTok{  }\NormalTok{Nutrients.processed }\OperatorTok
\StringTok{  }\KeywordTok{drop_na}\NormalTok{() }\OperatorTok
\StringTok{  }\KeywordTok{group_by}\NormalTok{(lakename) }\OperatorTok
\StringTok{  }\KeywordTok{summarize}\NormalTok{(}\DataTypeTok{meanTN =} \KeywordTok{mean}\NormalTok{(tn_ug), }
            \DataTypeTok{minTN =} \KeywordTok{min}\NormalTok{(tn_ug), }
            \DataTypeTok{maxTN =} \KeywordTok{max}\NormalTok{(tn_ug), }
            \DataTypeTok{sdTN =} \KeywordTok{sd}\NormalTok{(tn_ug))}

\NormalTok{Phosphorous.summary <-}\StringTok{ }
\StringTok{  }\NormalTok{Nutrients.processed }\OperatorTok
\StringTok{  }\KeywordTok{drop_na}\NormalTok{() }\OperatorTok
\StringTok{  }\KeywordTok{group_by}\NormalTok{(lakename) }\OperatorTok
\StringTok{  }\KeywordTok{summarize}\NormalTok{(}\DataTypeTok{meanTP =} \KeywordTok{mean}\NormalTok{(tp_ug), }
            \DataTypeTok{minTP =} \KeywordTok{min}\NormalTok{(tp_ug), }
            \DataTypeTok{maxTP =} \KeywordTok{max}\NormalTok{(tp_ug), }
            \DataTypeTok{sdTP =} \KeywordTok{sd}\NormalTok{(tp_ug))}
\end{Highlighting}
\end{Shaded}

Create a third R chunk that uses the function \texttt{kable} in the
knitr package to display two tables: one for the summary dataframe for
total N and one for the summary dataframe of total P. Use the
\texttt{caption\ =\ "\ "} code within that function to title your
tables. Customize the chunk options such that the final table is
displayed but not the code used to generate the table.

Create a fourth and fifth R chunk that generates two plots (one in each
chunk): one for total N over time with different colors for each lake,
and one with the same setup but for total P. Decide which geom option
will be appropriate for your purpose, and select a color palette that is
visually pleasing and accessible. Customize the chunk options such that
the final figures are displayed but not the code used to generate the
figures. In addition, customize the chunk options such that the figures
are aligned on the left side of the page. Lastly, add a fig.cap chunk
option to add a caption (title) to your plot that will display
underneath the figure.

\begin{Shaded}
\begin{Highlighting}[]
\KeywordTok{library}\NormalTok{(scales)}
\end{Highlighting}
\end{Shaded}

\begin{verbatim}
## 
## Attaching package: 'scales'
\end{verbatim}

\begin{verbatim}
## The following object is masked from 'package:purrr':
## 
##     discard
\end{verbatim}

\begin{verbatim}
## The following object is masked from 'package:readr':
## 
##     col_factor
\end{verbatim}

\begin{Shaded}
\begin{Highlighting}[]
  \KeywordTok{ggplot}\NormalTok{(Nutrients.processed, }\KeywordTok{aes}\NormalTok{(}\DataTypeTok{x =} \KeywordTok{as.POSIXct}\NormalTok{(sampledate), }\DataTypeTok{y =}\NormalTok{ tn_ug, }\DataTypeTok{color =}\NormalTok{ lakename)) }\OperatorTok{+}
\StringTok{  }\KeywordTok{geom_point}\NormalTok{() }\OperatorTok{+}
\StringTok{  }\KeywordTok{scale_color_brewer}\NormalTok{(}\DataTypeTok{palette=}\StringTok{"Dark2"}\NormalTok{) }\OperatorTok{+}
\StringTok{  }\KeywordTok{xlab}\NormalTok{(}\StringTok{"Date (year)"}\NormalTok{) }\OperatorTok{+}
\StringTok{  }\KeywordTok{ylab}\NormalTok{(}\StringTok{"Total Nitrogen (ug/L)"}\NormalTok{) }\OperatorTok{+}
\StringTok{  }\KeywordTok{scale_x_datetime}\NormalTok{(}\DataTypeTok{date_breaks =} \StringTok{"1 year"}\NormalTok{, }\DataTypeTok{labels =} \KeywordTok{date_format}\NormalTok{(}\StringTok{"%Y"}\NormalTok{))}
\end{Highlighting}
\end{Shaded}

\begin{verbatim}
## Warning: Removed 139 rows containing missing values (geom_point).
\end{verbatim}

\includegraphics{18_CraftingReports_files/figure-latex/Figure 1-1.pdf}
hello

\begin{Shaded}
\begin{Highlighting}[]
\KeywordTok{library}\NormalTok{(scales)}
\KeywordTok{ggplot}\NormalTok{(Nutrients.processed, }\KeywordTok{aes}\NormalTok{(}\DataTypeTok{x =} \KeywordTok{as.POSIXct}\NormalTok{(sampledate), }\DataTypeTok{y =}\NormalTok{ tp_ug, }\DataTypeTok{color =}\NormalTok{ lakename)) }\OperatorTok{+}
\StringTok{  }\KeywordTok{geom_point}\NormalTok{() }\OperatorTok{+}
\StringTok{  }\KeywordTok{scale_color_brewer}\NormalTok{(}\DataTypeTok{palette=}\StringTok{"Accent"}\NormalTok{) }\OperatorTok{+}
\StringTok{  }\KeywordTok{xlab}\NormalTok{(}\StringTok{"Date (year)"}\NormalTok{) }\OperatorTok{+}
\StringTok{  }\KeywordTok{ylab}\NormalTok{(}\StringTok{"Total Phosphorous (ug/L)"}\NormalTok{) }\OperatorTok{+}
\StringTok{  }\KeywordTok{scale_x_datetime}\NormalTok{(}\DataTypeTok{date_breaks =} \StringTok{"1 year"}\NormalTok{, }\DataTypeTok{labels =} \KeywordTok{date_format}\NormalTok{(}\StringTok{"%Y"}\NormalTok{))}
\end{Highlighting}
\end{Shaded}

\begin{verbatim}
## Warning: Removed 7 rows containing missing values (geom_point).
\end{verbatim}

\includegraphics{18_CraftingReports_files/figure-latex/unnamed-chunk-6-1.pdf}
\#\#\# Other options What are the chunk options that will suppress the
display of errors, warnings, and messages in the final document?

\begin{quote}
ANSWER: * suppress display of warnings - warnings=FALSE * suppress
display of messages - messages=FALSE
\end{quote}

\hypertarget{communicating-results}{%
\subsubsection{Communicating results}\label{communicating-results}}

Write a paragraph describing your findings from the R coding challenge
above. This should be geared toward an educated audience but one that is
not necessarily familiar with the dataset. Then insert a horizontal rule
below the paragraph. Below the horizontal rule, write another paragraph
describing the next steps you might take in analyzing this dataset. What
questions might you be able to answer, and what analyses would you
conduct to answer those questions?

\hypertarget{total-phosphorous-concentrations-experience-a-peak-between-1996-and-1997-where-east-long-lake-has-the-highest-concentrations-among-lakes-compared.-total-nitrogen-also-peaks-around-this-same-time-with-east-and-west-long-lakes-experiencing-the-highest-concentrations.}{%
\subsection{Total phosphorous concentrations experience a peak between
1996 and 1997, where East Long Lake has the highest concentrations among
lakes compared. Total nitrogen also peaks around this same time, with
East and West Long Lakes experiencing the highest
concentrations.}\label{total-phosphorous-concentrations-experience-a-peak-between-1996-and-1997-where-east-long-lake-has-the-highest-concentrations-among-lakes-compared.-total-nitrogen-also-peaks-around-this-same-time-with-east-and-west-long-lakes-experiencing-the-highest-concentrations.}}

Next steps for research might include gathering more data for other
contaminants included in the dataset (such as NH3\^{}4, NO2\^{}3, and
PO4). These variables have a lot of missing data, so gathering more data
would allow us to compare concentrations of total nitrogen and
phosphorous to the occurrence of these other chemical constituents. To
analyze these occurrences, a multiple linear regression could be used to
see if total nitrogen or phosphorous are predictors for the occurrence
of other compounds, or vice versa.

\hypertarget{knit-your-pdf}{%
\subsection{KNIT YOUR PDF}\label{knit-your-pdf}}

When you have completed the above steps, try knitting your PDF to see if
all of the formatting options you specified turned out as planned. This
may take some troubleshooting.

\hypertarget{other-r-markdown-customization-options}{%
\subsection{OTHER R MARKDOWN CUSTOMIZATION
OPTIONS}\label{other-r-markdown-customization-options}}

We have covered the basics in class today, but R Markdown offers many
customization options. A word of caution: customizing templates will
often require more interaction with LaTeX and installations on your
computer, so be ready to troubleshoot issues.

Customization options for pdf output include:

\begin{itemize}
\tightlist
\item
  Table of contents
\item
  Number sections
\item
  Control default size of figures
\item
  Citations
\item
  Template (more info
  \href{http://jianghao.wang/post/2017-12-08-rmarkdown-templates/}{here})
\end{itemize}

pdf\_document:\\
toc: true\\
number\_sections: true\\
fig\_height: 3\\
fig\_width: 4\\
citation\_package: natbib\\
template:

\end{document}
